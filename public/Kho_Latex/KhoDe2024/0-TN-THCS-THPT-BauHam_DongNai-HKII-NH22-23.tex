
\de{ĐỀ THI HỌC KỲ II NĂM HỌC 2022-2023}{THCS-THPT Bàu Hàm - Đồng Nai}
\begin{center}
	\textbf{PHẦN 1 - TRẮC NGHIỆM}
\end{center}
\Opensolutionfile{ans}[ans/ans]
\begin{ex}%[0X1B3-2]%[Dự án đề kiểm tra HKII NH22-23- Lương Như Quỳnh]%[Trường THCS-THPT Bàu Hàm]
Kết quả bài kiểm tra môn toán giữa kỳ 2 của tổ I lớp 10A được thống kê trong mẫu số liệu sau:
\[3,\,4,\,5,\,5,\,5,\,6,\,6,\,7,\,8,\,8,\,9.\]
Trung vị của mẫu số liệu trên là
\choice
{$8$}
{$5$}
{\True $6$}
{$4$}
\loigiai{
Trung vị của mẫu số liệu trên là $M_e=6$.\\
}
\end{ex}
\begin{ex}%[0X1B3-4]%[Dự án đề kiểm tra HKII NH22-23- Lương Như Quỳnh]%[Trường THCS-THPT Bàu Hàm]
Kết quả bài kiểm tra môn toán giữa kỳ 2 của tổ I lớp 10A được thống kê trong mẫu số liệu sau:
\[3,\,4,\,5,\,5,\,5,\,6,\,6,\,7,\,8,\,8,\,9.\]
Mốt của mẫu số liệu trên là
\choice
{$3$}
{\True $5$}
{$7$}
{$4$}
\loigiai{
Mốt của mẫu số liệu trên là $5$.
}
\end{ex}
\begin{ex}%[0X1B3-1]%[Dự án đề kiểm tra HKII NH22-23- Lương Như Quỳnh]%[Trường THCS-THPT Bàu Hàm]
Kết quả bài kiểm tra môn toán giữa kỳ 2 của tổ I lớp 10A được thống kê trong mẫu số liệu sau:
\[3,\,4,\,5,\,5,\,5,\,6,\,6,\,7,\,8,\,8,\,9.\]
Số trung bình của mẫu số liệu trên là
\choice
{$5{,}5$}
{$5$}
{\True $6$}
{$6{,}5$}
\loigiai{
Số trung bình của mẫu số liệu trên là $\overline{x}=\dfrac{3+4+5+5+5+6+6+7+8+8+9}{11}=6$.\\
}
\end{ex}
\begin{ex}%[0X1B4-1]%[Dự án đề kiểm tra HKII NH22-23- Lương Như Quỳnh]%[Trường THCS-THPT Bàu Hàm]
Kết quả bài kiểm tra môn toán giữa kỳ 2 của tổ I lớp 10A được thống kê trong mẫu số liệu sau:
\[3,\,4,\,5,\,5,\,5,\,6,\,6,\,7,\,8,\,8,\,9.\]
Các tứ phân vị của mẫu số liệu trên là
\choice
{\True $Q_1=5 $; $Q_2=6 $; $Q_3=8$}
{$Q_1=5 $; $Q_2=6 $; $Q_3=7$}
{$Q_1=3 $; $Q_2=4 $; $Q_3=5$}
{$Q_1=3 $; $Q_2=5 $; $Q_3=9$}
\loigiai{
Vì cỡ mẫu là $ n=11 $, là số lẻ, nên giá trị tứ phân vị thứ hai là $Q_2=6$.\\
Tứ phân vị thứ nhất là trung vị của mẫu: $ 3,\,4,\,5,\,5,\,5 $. Do đó $Q_1=5$.\\
Tứ phân vị thứ ba là trung vị của mẫu: $ 6,\,7,\,8,\,8,\,9$. Do đó $Q_3=8$.\\
Vậy các tứ phân vị của mẫu số liệu trên là $Q_1=5 $; $Q_2=6 $; $Q_3=8$.
}
\end{ex}
\begin{ex}%[0X1B4-1]%[Dự án đề kiểm tra HKII NH22-23- Lương Như Quỳnh]%[Trường THCS-THPT Bàu Hàm]
Kết quả bài kiểm tra môn toán giữa kỳ 2 của tổ I lớp 10A được thống kê trong mẫu số liệu sau:
\[3,\,4,\,5,\,5,\,5,\,6,\,6,\,7,\,8,\,8,\,9.\]
Khoảng biến thiên của mẫu số liệu trên là
\choice
{$11$}
{$9$}
{$3$}
{\True $6$}
\loigiai{
Trong mẫu số liệu trên, số lớn nhất là $ 9 $ và số nhỏ nhất là $ 3 $.\\
Vậy khoảng biến thiên của mẫu số liệu trên là $R=x_{\max}-x_{\min}=9-3=6$.
}
\end{ex}
\begin{ex}%[0X1B4-1]%[Dự án đề kiểm tra HKII NH22-23- Lương Như Quỳnh]%[Trường THCS-THPT Bàu Hàm]
Kết quả bài kiểm tra môn toán giữa kỳ 2 của tổ I lớp 10A được thống kê trong mẫu số liệu sau:
\[3,\,4,\,5,\,5,\,5,\,6,\,6,\,7,\,8,\,8,\,9.\]
Khoảng tứ phân vị của mẫu số liệu trên là
\choice
{$2$}
{$4$}
{\True $3$}
{$1$}
\loigiai{
Khoảng tứ phân vị của mẫu số liệu trên là $\Delta_Q=Q_3-Q_1=8-5=3$.
}
\end{ex}
\begin{ex}%[0X1B4-3]%[Dự án đề kiểm tra HKII NH22-23- Lương Như Quỳnh]%[Trường THCS-THPT Bàu Hàm]
Kết quả bài kiểm tra môn toán giữa kỳ 2 của tổ I lớp 10A được thống kê trong mẫu số liệu sau:
\[3,\,4,\,5,\,5,\,5,\,6,\,6,\,7,\,8,\,8,\,9.\]
Phương sai của mẫu số liệu trên là
\choice
{\True $\dfrac{34}{11}$}
{$\dfrac{34}{7}$}
{$\dfrac{11}{34}$}
{$\dfrac{7}{34}$}
\loigiai{
Số trung bình của mẫu số liệu trên là $\overline{x}=\dfrac{3+4+5+5+5+6+6+7+8+8+9}{11}=6$.\\
Phương sai của mẫu là \[S^2=\dfrac{1}{11}\left(3^2+4^2+5^2+5^2+5^2+6^2+6^2+7^2+8^2+8^2+9^2\right)-6^2=\dfrac{34}{11}.\]
}
\end{ex}
\begin{ex}%[0X1B4-3]%[Dự án đề kiểm tra HKII NH22-23- Lương Như Quỳnh]%[Trường THCS-THPT Bàu Hàm]
Kết quả bài kiểm tra môn toán giữa kỳ 2 của tổ I lớp 10A được thống kê trong mẫu số liệu sau:
\[3,\,4,\,5,\,5,\,5,\,6,\,6,\,7,\,8,\,8,\,9.\]
Độ lệch chuẩn của mẫu số liệu trên là
\choice
{\True $\sqrt{\dfrac{34}{11}}$}
{$\sqrt{\dfrac{34}{7}}$}
{$\sqrt{\dfrac{11}{34}}$}
{$\sqrt{\dfrac{7}{34}}$}
\loigiai{
Độ lệch chuẩn của mẫu số liệu trên là $S=\sqrt{\dfrac{34}{11}}$.
}
\end{ex}
\begin{ex}%[0X1B1-3]%[Dự án đề kiểm tra HKII NH22-23- Lương Như Quỳnh]%[Trường THCS-THPT Bàu Hàm]
Viết số quy tròn của số gần đúng $97\,637\,456$ với độ chính xác $d=300$.
\choice
{$97\,637\,500$}
{$97\,637\,756$}
{\True $97\,637\,000$}
{$97\,638\,000$}
\loigiai{
Hàng của chữ số khác $0$ đầu tiên bên trái của độ chính xác $d=300$ là hàng trăm, nên ta quy tròn số gần đúng $97\,637\,456$ đến hàng nghìn.\\
Vậy số quy tròn của $97\,637\,456$ là $97\,637\,000$.
}
\end{ex}
\begin{ex}%[0H4Y1-1]%[Dự án đề kiểm tra HKII NH22-23- Lương Như Quỳnh]%[Trường THCS-THPT Bàu Hàm]
Một véc-tơ pháp tuyến của đường thẳng $(d)\colon x-2 y+2023=0$ là
\choice
{$\overrightarrow{n}=(1;2)$}
{$\overrightarrow{n}=(2;1)$}
{$\overrightarrow{n}=(-2;1)$}
{\True $\overrightarrow{n}=(1;-2)$}
\loigiai{
Một véc-tơ pháp tuyến của đường thẳng $(d)\colon x-2y+2023=0$ là $\overrightarrow{n}=(1;-2)$.
}
\end{ex}

\begin{ex}%[0X3B1-1]%[Dự án đề kiểm tra HKII NH22-23- Lương Như Quỳnh]%[Trường THCS-THPT Bàu Hàm]
Tung một đồng xu liên tiếp hai lần. Không gian mẫu là
\choice
{\True $\Omega=\{SN; SS; NS; NN\}$}
{$\Omega=4$}
{$\Omega=\{SN; SS; NN\}$}
{$\Omega=3$}
\loigiai{
Không gian mẫu là $\Omega=\{SN; SS; NS; NN\}$.
}
\end{ex}

\begin{ex}%[0X3B1-1]%[Dự án đề kiểm tra HKII NH22-23- Lương Như Quỳnh]%[Trường THCS-THPT Bàu Hàm]
Tung một con xúc xắc liên tiếp hai lần. Số phần tử của không gian mẫu là
\choice
{$12$}
{$24$}
{\True $36$}
{$48$}
\loigiai{
Số phần tử của không gian mẫu là $6\cdot 6=36$.
}
\end{ex}
\begin{ex}%[0X3B1-1]%[Dự án đề kiểm tra HKII NH22-23- Lương Như Quỳnh]%[Trường THCS-THPT Bàu Hàm]
Tung một đồng xu liên tiếp $3$ lần. Số phần tử của không gian mẫu là
\choice
{\True $8$}
{$12$}
{$36$}
{$16$}
\loigiai{
Số phần tử của không gian mẫu là $2\cdot 2\cdot 2=8$.
}
\end{ex}
\begin{ex}%[0X3B2-4]%[Dự án đề kiểm tra HKII NH22-23- Lương Như Quỳnh]%[Trường THCS-THPT Bàu Hàm]
Từ một bộ bài tây $52$ lá, lấy ngẫu nhiên một lá. Xác suất để lấy được lá Át là
\choice
{$\dfrac{1}{4}$}
{$\dfrac{1}{52}$}
{$\dfrac{1}{26}$}
{\True $\dfrac{1}{13}$}
\loigiai{
Bộ bài tây có $4$ lá Át.\\
Xác suất để lấy được lá Át là $\dfrac{4}{52}=\dfrac{1}{13}$.
}
\end{ex}
\begin{ex}%[0X3B2-4]%[Dự án đề kiểm tra HKII NH22-23- Lương Như Quỳnh]%[Trường THCS-THPT Bàu Hàm]
Một hộp gồm các thẻ ghi số $1 $; $2 $; $3 $; $4 $; $5 $; $8 $; $9$. Lấy ngẫu nhiên một thẻ trong hộp. Xác suất để lấy được một thẻ ghi số chẵn là
\choice
{$\dfrac{1}{3}$}
{$\dfrac{1}{2}$}
{$\dfrac{2}{5}$}
{\True $\dfrac{3}{7}$}
\loigiai{
Số phần tử của không gian mẫu là $ 7 $.\\
Có $ 3 $ thẻ ghi số chẵn là $2 $; $4 $; $8 $.\\
Xác suất để lấy được một thẻ ghi số chẵn là $\dfrac{3}{7}$.
}
\end{ex}
\begin{ex}%[0X2B2-3]%[Dự án đề kiểm tra HKII NH22-23- Lương Như Quỳnh]%[Trường THCS-THPT Bàu Hàm]
Từ hộp chứa $5$ quả cầu trắng, $4$ quả cầu xanh kích thước và khối lượng giống nhau. Có bao nhiêu cách lấy ngẫu nhiên $3$ quả có đủ hai màu?
\choice
{\True $70$}
{$1200$}
{$25$}
{$10$}
\loigiai{
Số cách lấy ngẫu nhiên $2$ quả cầu trắng, $1$ quả cầu xanh là $ \mathrm{C}_5^2 \cdot 4=20$.\\
Số cách lấy ngẫu nhiên $1$ quả cầu trắng, $2$ quả cầu xanh là $ 5\cdot\mathrm{C}_4^2=50 $.\\
Vậy số cách lấy ngẫu nhiên $3$ quả có đủ hai màu là $70$.
}
\end{ex}
\begin{ex}%[0X3B2-4]%[Dự án đề kiểm tra HKII NH22-23- Lương Như Quỳnh]%[Trường THCS-THPT Bàu Hàm]
Một hộp chứa $11$ quả cầu gồm $5$ quả màu xanh, $4$ quả màu đỏ và $2$ quả màu vàng có kích thước và khối lượng giống hệt nhau. Lấy ngẫu nhiên đồng thời $3$ quả cầu từ hộp đó. Xác suất để $3$ quả cầu lấy ra có đủ $3$ màu là
\choice
{$\dfrac{8}{11}$}
{\True $\dfrac{8}{33}$}
{$\dfrac{1}{15}$}
{$\dfrac{5}{33}$}
\loigiai{
Số phần tử của không gian mẫu là $ \mathrm{C}_{11}^3 =165$.\\
Số cách lấy ngẫu nhiên $3$ quả cầu có đủ $3$ màu là $ 2\cdot 4 \cdot 5=40 $.\\
Xác suất để $3$ quả cầu lấy ra có đủ $3$ màu là $\dfrac{40}{165}=\dfrac{8}{33}$.
}
\end{ex}
\begin{ex}%[0X2B2-1]%[Dự án đề kiểm tra HKII NH22-23- Lương Như Quỳnh]%[Trường THCS-THPT Bàu Hàm]
Một đội tuyển học sinh giỏi gồm $3$ nam và $3$ nữ đứng ngẫu nhiên thành hàng ngang để chụp ảnh lưu niệm. Có bao nhiêu cách xếp sao cho nam và nữ đứng xen kẽ?
\choice
{$36$}
{\True $72$}
{$720$}
{$120$}
\loigiai{
Số cách xếp sao cho nam và nữ đứng xen kẽ là $3!\cdot 3!\cdot 2=72$.
}
\end{ex}
\begin{ex}%[0X3B2-4]%[Dự án đề kiểm tra HKII NH22-23- Lương Như Quỳnh]%[Trường THCS-THPT Bàu Hàm]
Một lô bóng đèn $24$ bóng, trong đó có $18$ bóng tốt và $6$ bóng kém chất lượng. Chọn ngẫu nhiên $2$ bóng. Xác suất để chọn được $2$ bóng kém chất lượng là
\choice
{$\dfrac{1}{3}$}
{$\dfrac{1}{4}$}
{\True $\dfrac{5}{92}$}
{$\dfrac{3}{4}$}
\loigiai{
Số phần tử của không gian mẫu là $ \mathrm{C}_{24}^2 =276$.\\
Số cách lấy ngẫu nhiên $2$ bóng kém chất lượng là $ \mathrm{C}_6^2 =15$.\\
Xác suất để chọn được $2$ bóng kém chất lượng là $\dfrac{15}{276}=\dfrac{5}{92}$.
}
\end{ex}
\begin{ex}%[0X3B2-4]%[Dự án đề kiểm tra HKII NH22-23- Lương Như Quỳnh]%[Trường THCS-THPT Bàu Hàm]
Một hộp có $20$ thẻ được đánh số từ $1$ đến $20$. Lấy ngẫu nhiên một thẻ từ hộp này. Xác suất để lấy được thẻ có số ghi trên thẻ chia hết cho $5$ là
\choice
{$\dfrac{1}{20}$}
{\True $\dfrac{1}{5}$}
{$\dfrac{1}{4}$}
{$\dfrac{4}{5}$}
\loigiai{
Số phần tử của không gian mẫu là $ 20 $.\\
Có $ 4 $ thẻ có số ghi trên thẻ chia hết cho $5$.\\
Xác suất để lấy được thẻ có số ghi trên thẻ chia hết cho $5$ là $\dfrac{4}{20}=\dfrac{1}{5}$.
}
\end{ex}
\begin{ex}%[0H4B1-5]%[Dự án đề kiểm tra HKII NH22-23- Lương Như Quỳnh]%[Trường THCS-THPT Bàu Hàm]
Trong mặt phẳng $Oxy$, cho điểm $M\left(x_0;y_0\right)$ và đường thẳng $(\Delta)\colon ax+by+c=0$. Khoảng cách từ điểm $M$ đến $\Delta$ được tính bằng công thức nào sau đây?
\choice
{$\mathrm{d}(M, \Delta)=\dfrac{\left|ax_0+b y_0+c\right|}{\sqrt{a+b}}$}
{$\mathrm{d}(M, \Delta)=\dfrac{ax_0+by_0+c}{\sqrt{a^2+b^2}}$}
{\True $\mathrm{d}(M, \Delta)=\dfrac{\left|ax_0+by_0+c\right|}{\sqrt{a^2+b^2}}$}
{$\mathrm{d}(M, \Delta)=\dfrac{ax_0+by_0+c}{\sqrt{a^2+b^2}}$}
\loigiai{
Khoảng cách từ điểm $M$ đến $\Delta$ được tính bằng công thức $\mathrm{d}(M, \Delta)=\dfrac{\left|ax_0+by_0+c\right|}{\sqrt{a^2+b^2}}$.
}
\end{ex}
\begin{ex}%[0H4Y2-1]%[Dự án đề kiểm tra HKII NH22-23- Lương Như Quỳnh]%[Trường THCS-THPT Bàu Hàm]
Trong mặt phẳng với hệ tọa độ $Oxy$, cho đường tròn $(C)\colon x^2+y^2+2x-6y+5=0$. Tọa độ tâm $I$ của đường tròn $(C)$ là
\choice
{$I(2;-6)$}
{$I(1;-3)$}
{\True $I(-1;3)$}
{$I(1;3)$}
\loigiai{
Ta có $(C)\colon x^2+y^2+2x-6y+5=0 \Leftrightarrow x^2+y^2-2\cdot(-1)x-2\cdot 3y+5=0$.\\
Vậy tọa độ tâm $I$ của đường tròn $(C)$ là $I(-1;3)$.
}
\end{ex}
\begin{ex}%[0H4B1-5]%[Dự án đề kiểm tra HKII NH22-23- Lương Như Quỳnh]%[Trường THCS-THPT Bàu Hàm]
Trong mặt phẳng với hệ tọa độ $Oxy$, cho điểm $M(2;-3)$ và đường thẳng
$(\Delta)\colon 3x-4y-8=0$. Tính khoảng cách $d$ từ $M$ đến $(\Delta)$ là
\choice
{\True $d=2$}
{$d=5$}
{$d=10$}
{$d=16$}
\loigiai{
Khoảng cách $d$ từ $M$ đến $(\Delta)$ là
$\mathrm{d}(M, \Delta)=\dfrac{\left|3\cdot 2-4(-3)-8\right|}{\sqrt{3^2+(-4)^2}}=2$.
}
\end{ex}
\begin{ex}%[0H4Y2-2]%[Dự án đề kiểm tra HKII NH22-23- Lương Như Quỳnh]%[Trường THCS-THPT Bàu Hàm]
Phương trình chính tắc của đường tròn tâm $I(a;b)$, bán kính $R$ là
\choice
{\True $(x-a)^2+(y-b)^2=R^2$}
{$(x+a)^2+(y+b)^2=R^2$}
{$\left(x_0-a\right)(x-a)+\left(y_0-b\right)(y-b)=R^2$}
{$(x-a)^2+(y-b)^2=R$}
\loigiai{
Phương trình chính tắc của đường tròn tâm $I(a;b)$, bán kính $R$ là $(x-a)^2+(y-b)^2=R^2$.
}
\end{ex}
\begin{ex}%[0H4B2-3]%[Dự án đề kiểm tra HKII NH22-23- Lương Như Quỳnh]%[Trường THCS-THPT Bàu Hàm]
Trong mặt phẳng $Oxy$, cho đường thẳng $(\Delta)\colon 4x-3y-3=0$. Đường tròn tâm $I(1;-3)$ nào sau đây tiếp xúc với $(\Delta)$?
\choice
{$(x+1)^2+(y-3)^2=4$}
{\True $(x-1)^2+(y+3)^2=4$}
{$(x+1)^2+(y-3)^2=2$}
{$(x-1)^2+(y+3)^2=2$}
\loigiai{
Ta có đường tròn tâm $I(1;-3)$ tiếp xúc với $(\Delta)$ nên bán kính đường tròn là \[R=\mathrm{d}(I,\Delta)=\dfrac{\left|4-3(-3)-3\right|}{\sqrt{4^2+(-3)^2}}=2.\]
Phương trình đường tròn $(C)\colon (x-1)^2+(y+3)^2=4$.
}
\end{ex}
\begin{ex}%[0H4B1-3]%[Dự án đề kiểm tra HKII NH22-23- Lương Như Quỳnh]%[Trường THCS-THPT Bàu Hàm]
Trong mặt phẳng với hệ tọa độ $Oxy$, cho đường thẳng $(\Delta)\colon 2x-y+1=0$. Đường thẳng nào sau đây song song với $(\Delta)$?
\choice
{$2x+y-1=0$}
{$x+y-1=0$}
{$x+2y-1=0$}
{\True $4x-2y+3=0$}
\loigiai{
Đường thẳng $(\Delta)$ có một véc-tơ pháp tuyến là $\overrightarrow{n}=(2;-1)$.\\
Suy ra đường thẳng song song với $(\Delta)$ có dạng $2mx-my+c=0$, với $m\neq 0$, $c\neq m$.
}
\end{ex}
\begin{ex}%[0H4B2-2]%[Dự án đề kiểm tra HKII NH22-23- Lương Như Quỳnh]%[Trường THCS-THPT Bàu Hàm]
Trong mặt phẳng với hệ tọa độ $Oxy$, phương trình nào dưới đây là phương trình của đường tròn tâm $I(-2;1)$ và đi qua $A(1;2)$?
\choice
{$(x+2)^2+(y-1)^2=100$}
{$(x-2)^2+(y+1)^2=10$}
{$(x+2)^2+(y-1)^2=\sqrt{10}$}
{\True $(x+2)^2+(y-1)^2=10$}
\loigiai{
Ta có $ \overrightarrow{IA}=(3;1) $.\\
Đường tròn tâm $I(-2;1)$ và đi qua $A(1;2)$ có bán kính $ R=IA=\sqrt{10} $.\\
Phương trình đường tròn $(C)\colon(x+2)^2+(y-1)^2=10$.
}
\end{ex}
\begin{ex}%[0H4Y2-1]%[Dự án đề kiểm tra HKII NH22-23- Lương Như Quỳnh]%[Trường THCS-THPT Bàu Hàm]
Trong mặt phẳng tọa độ $Oxy$, tọa độ tâm $I$ và bán kính $R$ của đường tròn $(C)$ có phương trình $(x+2)^2+(y-4)^2=9$ là
\choice
{$I(2;-4)$, $R=3$}
{$I(-1;2)$, $R=3$}
{$I(1;-2)$, $R=3$}
{\True $I(-2;4)$, $R=3$}
\loigiai{
Đường tròn $(C)$ có tâm $I(-2;4)$, bán kính $R=3$.
}
\end{ex}
\begin{ex}%[0H4B2-3]%[Dự án đề kiểm tra HKII NH22-23- Lương Như Quỳnh]%[Trường THCS-THPT Bàu Hàm]
Trong mặt phẳng tọa độ $Oxy$, cho đường tròn $(C)$ có phương trình $(x-1)^2+(y+2)^2=8$. Phương trình tiếp tuyến $d$ của đường tròn $(C)$ tại điểm $A(3;-4)$ là
\choice
{$d\colon x+y+1=0$}
{$d\colon x-2y-11=0$}
{\True $d\colon x-y-7=0$}
{$d\colon x-y+7=0$}
\loigiai{
Đường tròn $(C)$ có tâm $I(1;-2)$.\\
Ta có $ \overrightarrow{IA}=(2;-2) =2\cdot(1;-1)$.\\
Tiếp tuyến $d$ của đường tròn $(C)$ tại điểm $A(3;-4)$ có một véc-tơ pháp tuyến là $ \overrightarrow{n}=(1;-1) $
Phương trình tiếp tuyến $d$ của đường tròn $(C)$ tại điểm $A(3;-4)$ là \[d\colon (x-3)-(y+4)=0\Leftrightarrow x-y-7=0.\]
}
\end{ex}
\begin{ex}%[0H4Y1-4]%[Dự án đề kiểm tra HKII NH22-23- Lương Như Quỳnh]%[Trường THCS-THPT Bàu Hàm]
Trong mặt phẳng $Oxy$, gọi $\varphi$ là góc tạo bởi hai đường thẳng $\Delta_1\colon a_1 x+b_1 y+c_1=0$ và $\Delta_2\colon a_2 x+b_2 y+c_2=0$. Khẳng định nào sau đây đúng?
\choice
{$\cos \varphi=\dfrac{a_1 a_2+b_1 b_2}{\sqrt{a_1^2+b_1^2} \cdot \sqrt{a_2^2+b_2^2}}$}
{\True $\cos \varphi=\dfrac{\left|a_1 a_2+b_1 b_2\right|}{\sqrt{a_1^2+b_1^2} \cdot \sqrt{a_2^2+b_2^2}}$}
{$\cos \varphi=\dfrac{\left|a_1 b_1+a_2 b_2\right|}{\sqrt{a_1^2+b_1^2} \cdot \sqrt{a_2^2+b_2^2}}$}
{$\cos \varphi=\dfrac{\left|a_1 b_2+a_2 b_1\right|}{\sqrt{a_1^2+b_1^2} \cdot \sqrt{a_2^2+b_2^2}}$}
\loigiai{
Ta có $\cos \varphi=\dfrac{\left|a_1 a_2+b_1 b_2\right|}{\sqrt{a_1^2+b_1^2} \cdot \sqrt{a_2^2+b_2^2}}$.
}
\end{ex}

%%=====Câu 31
\begin{ex}%[0H4B1-4]%[Dự án đề kiểm tra HKII NH22-23- Thầy Hóa]%[THCS-THPT Bàu Ham - Đồng Nai]
Trong mặt phẳng $Oxy$, cho hai đường thẳng $\left(\Delta_1\right)\colon x-2 y+3=0$ và $\left(\Delta_2\right)\colon x+3 y+6=0$. Số đo của góc giữa hai đường thẳng $\Delta_1$, $\Delta_2$ là
\choice
{\True $45^\circ$}
{$135^\circ$}
{$30^\circ$}
{$60^\circ$}
\loigiai{
Đường thẳng $\Delta_1$ có một véc-tơ pháp tuyến là $\vec{n}_1=(1;-2)$.\\
Đường thẳng $\Delta_2$ có một véc-tơ pháp tuyến là $\vec{n}_2=(1;3)$.\\
Gọi $\alpha$ là góc giữa hai đường thẳng $\Delta_1$ và $\Delta_2$. Khi đó, ta có
\allowdisplaybreaks
\begin{eqnarray*}
	\cos \alpha&=&\dfrac{\left|\vec{n}_1\cdot \vec{n}_2\right|}{\left|\vec{n}_1\right|\cdot \left|\vec{n}_2\right|}\\
	&=&\dfrac{\left|1\cdot 1+(-2)\cdot 3\right|}{\sqrt{1^2+(-2)^2}\cdot \sqrt{1^2+3^2}}\\
	&=&\dfrac{1}{\sqrt{2}}.
\end{eqnarray*}
Suy ra $\alpha=45^\circ$.
}
\end{ex}

%%=====Câu 32
\begin{ex}%[0H4B3-1]%[Dự án đề kiểm tra HKII NH22-23- Thầy Hóa]%[THCS-THPT Bàu Ham - Đồng Nai]
Trong mặt phẳng tọa độ $Oxy$, cho Elip $(E)\colon\dfrac{x^2}{9}+\dfrac{y^2}{4}=1$. Độ dài trục lớn của $(E)$ là
\choice
{\True $6$}
{$3$}
{$2$}
{$4$}
\loigiai{
Ta có $a^2=9\Rightarrow a=\pm 3$.\\
Độ dài trục lớn của $(E)$ là $2a=2\cdot 3=6$.
}
\end{ex}

%%=====Câu 33
\begin{ex}%[0H4K3-2]%[Dự án đề kiểm tra HKII NH22-23- Thầy Hóa]%[THCS-THPT Bàu Ham - Đồng Nai]
Trong mặt phẳng tọa độ $Oxy$, cho Elip có một tiêu điểm $F_1(-3;0)$ và đi qua điểm $K(0;-4)$. Phương trình chính tắc của Elip đó là
\choice
{$\dfrac{x^2}{5}+\dfrac{y^2}{4}=1$}
{\True $\dfrac{x^2}{25}+\dfrac{y^2}{16}=1$}
{$\dfrac{x^2}{25}+\dfrac{y^2}{16}=-1$}
{$\dfrac{x^2}{25}-\dfrac{y^2}{16}=1$}
\loigiai{
Giả sử phương trình chính tắc của Elip có dạng $\dfrac{x^2}{a^2}+\dfrac{y^2}{b^2}=1$.\\
Vì Elip đi qua điểm $K(0;-4)$ nên
\[\dfrac{0^2}{a^2}+\dfrac{(-4)^2}{b^2}=1\Leftrightarrow b^2=16. \]
Mặt khác Elip có một tiêu điểm $F_1(-3;0)$ nên $c=3$.\\
Mà $a^2-b^2=c^2$ nên $a^2-16=9\Leftrightarrow a^2=25$.\\
Vậy phương trình chính tắc của Elip là $\dfrac{x^2}{25}+\dfrac{y^2}{16}=1$.
}
\end{ex}

%%=====Câu 34
\begin{ex}%[0H4B3-1]%[Dự án đề kiểm tra HKII NH22-23- Thầy Hóa]%[THCS-THPT Bàu Ham - Đồng Nai]
Trong mặt phẳng tọa độ $Oxy$, cho Elip $(E)\colon\dfrac{x^2}{169}+\dfrac{y^2}{144}=1$. Tiêu cự của $(E)$ là
\choice
{$F_1F_2=26$}
{$F_1F_2=24$}
{\True $F_1F_2=10$}
{$F_1F_2=5$}
\loigiai{
Ta có $\heva{&a^2=169\\&b^2=144}\Rightarrow c^2=a^2-b^2=169-144=25\Leftrightarrow c=5$.\\
Vậy tiêu cự của $(E)$ là $F_1F_2=2c=2\cdot 5=10$.
}
\end{ex}

%%=====Câu 35
\begin{ex}%[0H4B2-2]%[Dự án đề kiểm tra HKII NH22-23- Thầy Hóa]%[THCS-THPT Bàu Ham - Đồng Nai]
Trong mặt phẳng tọa độ $Oxy$, cho ba điểm $A(-1;-4)$, $B(3;-4)$, $C(3;0)$. Đường tròn ngoại tiếp tam giác $ABC$ có phương trình là
\choice
{$x^2+y^2-2x+4y+3=0$}
{$x^2+y^2+2x+4y-3=0$}
{$x^2+y^2-2x-4y-3=0$}
{\True $x^2+y^2-2x+4y-3=0$}
\loigiai{
Gọi $I(a;b)$ là tâm của đường tròn ngoại tiếp tam giác $ABC$, ta có
\allowdisplaybreaks
\begin{eqnarray*}
	\heva{&AI^2=BI^2\\&AI^2=CI^2}&\Leftrightarrow &\heva{&(a+1)^2+(b+4)^2=(a-3)^2+(b+4)^2\\&(a+1)^2+(b+4)^2=(a-3)^2+b^2}\\
	&\Leftrightarrow &\heva{&a^2+2a+1=a^2-6a+9\\&a^2+2a+1+b^2+8b+16=a^2-6a+9+b^2}\\
	&\Leftrightarrow &\heva{&8a=8\\&8a+8b=-8}\\
	&\Leftrightarrow &\heva{&a=1\\&b=-2.}
\end{eqnarray*}
Suy ra $I(1;-2)$, suy ra $R^2=AI^2=(1+1)^2+(-2+4)^2=8$.\\
Phương trình đường tròn ngoại tiếp tam giác $ABC$ là
\[(x-1)^2+(y+2)^2=8\Leftrightarrow x^2+y^2-2x+4y-3=0. \]
}
\end{ex}




\Closesolutionfile{ans}
%\begin{center}
%	\textbf{ĐÁP ÁN}
%	\inputansbox{10}{ans/ans}	
%\end{center}
\begin{center}
	\textbf{PHẦN 2 - TỰ LUẬN}
\end{center}


%%=====Bài 1
\begin{bt}%[0X3B2-3]%[Dự án đề kiểm tra HKII NH22-23- Thầy Hóa]%[THCS-THPT Bàu Ham - Đồng Nai]
Một đội gồm $3$ nam và $6$ nữ, người ta muốn lập ngẫu nhiên một tổ công tác gồm $3$ người. Tính xác suất để tổ công tác đó có đúng $1$ nam.
\loigiai{
Ta có số phần tử của không gian mẫu $n(\Omega)=\mathrm{C}_9^3$.\\
Gọi biến cố $A$: ``Trong 3 người được chọn có 1 nam, 2 nữ''.\\
Suy ra $n(A)=\mathrm{C}_3^1\cdot \mathrm{C}_6^2$.\\
Xác suất để tổ công tác đó có đúng 1 nam là
\[P(A)=\dfrac{n(A)}{n(\Omega)}=\dfrac{\mathrm{C}_3^1\cdot \mathrm{C}_6^2}{\mathrm{C}_9^3}=\dfrac{15}{28}. \]
}
\end{bt}

%%=====Bài 2
\begin{bt}%[0H4B2-2]%[Dự án đề kiểm tra HKII NH22-23- Thầy Hóa]%[THCS-THPT Bàu Ham - Đồng Nai]
Trong mặt phẳng tọa độ $Oxy$, viết phương trình đường tròn $(C)$, biết
\begin{enumerate}
	\item $(C)$ có tâm $I(3;-4)$ bán kính $R=\sqrt{3}$.
	\item $(C)$ có tâm $I(3;-4)$ và tiếp xúc với đường thẳng $(\Delta)\colon x-2y-6=0$.
\end{enumerate}
\loigiai{
\begin{enumerate}
	\item Phương trình đường tròn $(C)$ có tâm $I(3;-4)$ bán kính $R=\sqrt{3}$ là
	\[(C)\colon (x-3)^2+(y+4)^2=3. \] 
	\item Vì $(C)$ tiếp xúc với đường thẳng $(\Delta)\colon x-2y-6=0$ nên
	\[R=\mathrm{d}(I, \Delta)=\dfrac{|3+8-6|}{\sqrt{5}}=\sqrt{5}. \]
	Do đó phương trình đường tròn $(C)$ có tâm $I(3;-4)$ bán kính $R=\sqrt{5}$ là
	\[(C)\colon (x-3)^2+(y+4)^2=5. \]
\end{enumerate}
}
\end{bt}

%%=====Bài 3
\begin{bt}%[0X3K2-6]%[Dự án đề kiểm tra HKII NH22-23- Thầy Hóa]%[THCS-THPT Bàu Ham - Đồng Nai]
Một hộp chứa $15$ viên bi gồm $6$ viên màu đỏ được đánh số từ $1$ đến $6$ và $9$ viên màu xanh được đánh số từ $1$ đến $9$ (các viên bi chỉ khác nhau về màu sắc). Lấy ngẫu nhiên $2$ viên bi từ hộp này. Tính xác suất để lấy được hai viên bi khác màu và tổng hai số ghi trên chúng là số chẵn.
\loigiai{
Ta có số phần tử của không gian mẫu $n(\Omega)=\mathrm{C}_{15}^2=105$.\\
Gọi biến cố $A$: ``Hai viên bi được lấy ra khác màu và tổng hai số ghi trên chúng là số chẵn''. Ta có 2 trường hợp sau
\begin{itemize}
	\item[TH1:] Bi xanh và bi đỏ cùng ghi số lẻ, số cách chọn là $\mathrm{C}_5^1\cdot \mathrm{C}_3^1=15$.
	\item[TH2:] Bi xanh và bi đỏ cùng ghi số chẵn, số cách chọn là $\mathrm{C}_4^1\cdot \mathrm{C}_3^1=12$.
\end{itemize}
Theo quy tắc cộng, ta được $n(A)=15+12=27$.\\
Xác suất của biến cố $A$ là
\[P(A)=\dfrac{n(A)}{n(\Omega)}=\dfrac{27}{105}=\dfrac{9}{35}. \]
}
\end{bt}

%%=====Bài 4
\begin{bt}%[0H4G1-7]%[Dự án đề kiểm tra HKII NH22-23- Thầy Hóa]%[THCS-THPT Bàu Ham - Đồng Nai]
Trong mặt phẳng tọa độ $Oxy$, cho $\triangle ABC$ có $A(2; 3)$, $B(3;-1)$, $C(3; 2)$. Viết phương trình đường thẳng $(d)$ đi qua $A$ và chia $\triangle ABC$ thành hai phần sao cho phần chứa cạnh $AB$ có diện tích gấp hai lần phần chứa cạnh $AC$.
\loigiai{
\immini{
Gọi $M$ là giao điểm của đường thẳng $d$ với $BC$.\\
Vì đường thẳng $(d)$ đi qua $A$ và chia $\triangle ABC$ thành hai phần sao cho phần chứa cạnh $AB$ có diện tích gấp hai lần phần chứa cạnh $AC$ nên
\[S_{\triangle MAB}=2S_{\triangle MAC}\Leftrightarrow MB=2MC. \]
Suy ra $\begin{aligned}[t]
	\vec{BM}=\dfrac{2}{3}\vec{BC}&\Leftrightarrow \heva{&x_M-3=\dfrac{2}{3}(3-3)\\&y_M+1=\dfrac{2}{3}(2+1)}\\
	&\Leftrightarrow \heva{&x_M=3\\&y_M=1}\Rightarrow M(3;1).
\end{aligned}$
}{
\begin{tikzpicture}[>=stealth,line join=round,line cap=round,font=\footnotesize,scale=1]
	\path
	(2,3) coordinate (A)
	(3,-1) coordinate (B)
	(3,2) coordinate (C)
	($(B)!(A)!(C)$) coordinate (H)
	($(B)!2/3!(C)$) coordinate (M)
	;
	\draw
	(A)--(B)--(H)--cycle (C)--(A)--(M)
	;
	\foreach \x/\g in {B/-90,A/135,C/0,H/45,M/0}\fill[black](\x) circle (1pt) +(\g:3mm) node {$\x$};
	\foreach \x/\y/\z in {A/H/C}{\draw pic[draw,angle radius=2mm]{right angle=\x--\y--\z};}
\end{tikzpicture}
}
\noindent Do đó đường thẳng $d$ đi qua hai điểm $A$ và $M$.\\
Ta có véc-tơ chỉ phương $\vec{AM}=(1;-2)$.\\
Suy ra véc-tơ pháp tuyến $\vec{n}=(2;1)$.\\
Phương trình tổng quát của đường thẳng $d$ là
\[(d)\colon2\cdot (x-3)+1\cdot (y-1)=0\Leftrightarrow 2x+y-7=0. \]
}
\end{bt}
