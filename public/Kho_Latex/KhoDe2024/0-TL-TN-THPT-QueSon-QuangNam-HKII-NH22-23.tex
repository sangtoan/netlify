
\de{ĐỀ THI HỌC KỲ II NĂM HỌC 2022-2023}{THPT Quế Sơn - Quảng Nam}
\begin{center}
	\textbf{PHẦN 1 - TRẮC NGHIỆM}
\end{center}
\Opensolutionfile{ans}[ans/ans]
%Câu 1...........................
\begin{ex}%[0D3B1-2]%[Dự án đề kiểm tra HKII NH22-23- Huỳnh Quy]%[THPT Quế Sơn, Quảng Nam]
		Tìm tập xác định $\mathscr{D}$ của hàm số $y=\sqrt{x-1}$.
		\choice
		{$\mathscr{D}=\mathbb{R}\setminus\{1\}$}
		{$\mathscr{D}=(1;+\infty)$}
		{\True $\mathscr{D}=[1;+\infty)$}
		{$\mathscr{D}=(-\infty;1]$}
	\loigiai{
	Hàm số xác định $\Leftrightarrow x-1\geq 0\Leftrightarrow x\geq 1$.\\
	Vậy tập xác định là $\mathscr{D}=[1;+\infty)$.	
	}
\end{ex}

\begin{ex}%[0D3B2-1]%[Dự án đề kiểm tra HKII NH22-23- Huỳnh Quy]%[THPT Quế Sơn, Quảng Nam]%[cau 2]
	Toạ độ đỉnh $I$ của parabol $(P)\colon y=x^2+2x+4$ là
	\choice
	{$I(2;5)$}
	{\True $I(-1;3)$}
	{$I(-1;1)$}
	{$I(1;7)$}
	\loigiai{
	Toạ độ đỉnh $I$ là $\heva{&x_I=\dfrac{-b}{2a}=\dfrac{-2}{2\cdot 1}=-1\\&y_I=(-1)^2+2\cdot(-1)+4=3.}$	
	}
\end{ex}

\begin{ex}%[0D4B1-2]%[Dự án đề kiểm tra HKII NH22-23- Huỳnh Quy]%[THPT Quế Sơn, Quảng Nam]%[cau 3]
	Cho tam thức bậc hai $f(x)=ax^2+bx+c$ với $a>0$ và $\Delta=b^2-4ac$. Phát biểu nào sau đây đúng?
	\choice
	{Nếu $\Delta>0$ thì $f(x)>0$, $\forall x\in\mathbb{R}$}
	{Nếu $\Delta>0$ thì $f(x)<0$, $\forall x\in\mathbb{R}$}
	{\True Nếu $\Delta<0$ thì $f(x)>0$, $\forall x\in\mathbb{R}$}
	{Nếu $\Delta<0$ thì $f(x)<0$, $\forall x\in\mathbb{R}$}
	\loigiai{
	Phát biểu đúng là ``Nếu $\Delta<0$ thì $f(x)>0$, $\forall x\in\mathbb{R}$''.	
	}
\end{ex}

\begin{ex}%[0D4B3-2]%[Dự án đề kiểm tra HKII NH22-23- Huỳnh Quy]%[THPT Quế Sơn, Quảng Nam]%[cau 4]
	Tính tổng $S$ các nghiệm của phương trình $\sqrt{3x-8}=x-2$.
	\choice
	{\True $S=7$}
	{$S=-7$}
	{$S=-12$}
	{$S=12$}
	\loigiai{
	Xét phương trình $\sqrt{3x-8}=x-2$, bình phương hai vế phương trình ta được:
	\begin{eqnarray*}
	&&\sqrt{3x-8}=x-2\\
	&\Rightarrow&3x-8=(x-2)^{2}\\
	&\Rightarrow&3x-8=x^2-4x+4\\
	&\Rightarrow&x^2-7x+12=0\\
	&\Rightarrow&\hoac{&x=4\\&x=3.}
	\end{eqnarray*}
	Thử lại, ta thấy cả hai giá trị $x=3$, $x=4$ đều thoả mãn.\\
	Do đó, phương trình có hai nghiệm $x=3$, $x=4$.\\
	Tổng các nghiệm là $3+4=7$.
	}
\end{ex}

\begin{ex}%[0H4B1-1]%[Dự án đề kiểm tra HKII NH22-23- Huỳnh Quy]%[THPT Quế Sơn, Quảng Nam]%[cau 5]
	Trong mặt phẳng $Oxy$, cho đường thẳng $d\colon\heva{&x=-1+3t\\&y=5-2t}$. Véc tơ nào sau đây là véc-tơ chỉ phương của đường thẳng $d$?
	\choice
	{$\vec{u}_1=(2;3)$}
	{$\vec{u}_2=(3;2)$}
	{$\vec{u}_3=(-1;5)$}
	{\True $\vec{u}_4=(3;-2)$}
	\loigiai{
	Véc-tơ chỉ phương của đường thẳng $d$ là $\vec{u}_4=(3;-2)$.	
	}
\end{ex}
\begin{ex}%[0H4B1-4]%[Dự án đề kiểm tra HKII NH22-23- Huỳnh Quy]%[THPT Quế Sơn, Quảng Nam]%[cau 6]
	Góc giữa hai đường thẳng $\Delta_1\colon 3x+y+5=0$ và $\Delta_2\colon -2x+y-7=0$ bằng
	\choice
	{$30^{\circ}$}
	{$60^{\circ}$}
	{\True $45^{\circ}$}
	{$135^{\circ}$}
	\loigiai{
	$\Delta_1$ có véc-tơ pháp tuyến $\vec{n}_1=(3;1)$;\\
	$\Delta_2$ có véc-tơ pháp tuyến $\vec{n}_2=(-2;1)$.\\
	Ta có
	\begin{eqnarray}
	\cos(\Delta_1,\Delta_2)&=&=\dfrac{|\vec{n}_1\cdot\vec{n}_2|}{|\vec{n}_1|\cdot|\vec{n}_{2}|}\\
	&=&\dfrac{|3\cdot(-2)+1\cdot 1|}{\sqrt{3^2+1^2}\cdot\sqrt{(-2)^2+1^2}}\\
	&=&\dfrac{5\sqrt{10}\cdot\sqrt{5}}{\dfrac{\sqrt{2}}{2}}\\
	&\Rightarrow&(\Delta_1;\Delta_2)=45^{\circ}.
	\end{eqnarray}
	}
\end{ex}

\begin{ex}%[0H4B2-1]%[Dự án đề kiểm tra HKII NH22-23- Huỳnh Quy]%[THPT Quế Sơn, Quảng Nam]%[cau 7]
	Trong mặt phẳng $Oxy$, cho đường tròn $(C)\colon x^2+y^2+4x-6y-3=0$. Tìm toạ độ tâm $I$ và tính bán kính $R$ của $(C)$.
	\choice
	{$I(-2;3)$, $R=16$}
	{\True $I(-2;3)$, $R=4$}
	{$I(2;-3)$, $R=15$}
	{$I(2;-3)$, $R=4$}
	\loigiai{
	Đường tròn $(C)$ có tâm $I(-2;3)$, bán kính $R=\sqrt{(-2)^2+3^2-(-3)}=4$.	
	}
\end{ex}

\begin{ex}%[0H4B3-1]%[Dự án đề kiểm tra HKII NH22-23- Huỳnh Quy]%[THPT Quế Sơn, Quảng Nam]%[cau 8]
	Trong mặt phẳng $Oxy$, cho elip $(E)$ có phương trình chính tắc $\dfrac{x^2}{100}+\dfrac{y^2}{64}=1$. Tiêu cự của elip bằng
	\choice
	{$16$}
	{$20$}
	{\True $12$}
	{$4\sqrt{41}$}
	\loigiai{
	Ta có $\heva{&a^2=100\Rightarrow a=10\\&b^2=64\Rightarrow b=8.}$\\
	$\Rightarrow c=\sqrt{a^2-b^2}=\sqrt{10^2-8^2}=6$.\\
	Tiêu cự là $2c=12$.	
	}
\end{ex}

\begin{ex}%[0D2B1-1]%[Dự án đề kiểm tra HKII NH22-23- Huỳnh Quy]%[THPT Quế Sơn, Quảng Nam]%[cau 9]
	Lớp $10A$ có $15$ học sinh nam và $20$ học sinh nữ. Hỏi có bao nhiêu cách chọn ra một học sinh để tham gia vào đội thanh niên tình nguyện của trường, biết rằng tất cả các bạn trong lớp đều có khả năng tham gia.
	\choice
	{$\mathrm{C}_{29}^{15}$}
	{\True $35$}
	{$300$}
	{$20$}
	\loigiai{
	Số cách chọn ra $1$ học sinh là $15+20=35$ cách.	
	}
\end{ex}

\begin{ex}%[0D2B1-2]%[Dự án đề kiểm tra HKII NH22-23- Huỳnh Quy]%[THPT Quế Sơn, Quảng Nam]%[cau 10]
	Có tất cả bao nhiêu số tự nhiên gồm $3$ chữ số?
	\choice
	{$\mathrm{A}_{10}^{3}$}
	{\True $900$}
	{$\mathrm{C}_{ 10}^{3}$}
	{$\mathrm{C}_{9}^{3}$}
	\loigiai{
	Gọi $n=\overline{abc}$ là số cần lập.
	\begin{itemize}
		\item $a\ne 0$ nên $a$ có $9$ cách chọn.
		\item $b$ có $10$ cách chọn.
		\item $c$ có $10$ cách chọn.
	\end{itemize}	
Vậy có tất cả $9\cdot 10\cdot 10=900$ cách.
	}
\end{ex}

\begin{ex}%[0D2B2-1]%[Dự án đề kiểm tra HKII NH22-23- Huỳnh Quy]%[THPT Quế Sơn, Quảng Nam]%[cau 11]
	Có bao nhiêu cách sắp xếp $4$ học sinh thành một hàng dọc?
	\choice
	{$\mathrm{C}_{4}^{1}$}
	{$16$}
	{$8$}
	{\True $24$}
	\loigiai{
	Mỗi cách xếp thoả đề chính là một hoán vị của $4$ học sinh. Vậy có $4!=24$ cách	 xếp.
	}
\end{ex}

\begin{ex}%[0D2Y2-8]%[Dự án đề kiểm tra HKII NH22-23- Huỳnh Quy]%[THPT Quế Sơn, Quảng Nam]%[cau 12]
	Công thức tính số chỉnh hợp chập $k$ của $n$ phần tử là
	\choice
	{\True $\mathrm{A}_{n}^{k}=\dfrac{n!}{(n-k)!}$}
	{$\mathrm{A}_{n}^{k}=\dfrac{n!}{k!(n-k)!}$}
	{$\mathrm{A}_{n}^{k}=\dfrac{n!}{k!}$}
	{$\mathrm{A}_{n}^{k}=\dfrac{k!}{(n-k)!}$}
	\loigiai{
	Công thức tính số chỉnh hợp chập $k$ của $n$ phần tử là $\mathrm{A}_{n}^{k}=\dfrac{n!}{(n-k)!}$.	
	}
\end{ex}

\begin{ex}%[0D2B2-2]%[Dự án đề kiểm tra HKII NH22-23- Huỳnh Quy]%[THPT Quế Sơn, Quảng Nam]%[cau 13]
	Có $5$ viên bi màu xanh và $10$ viên bi màu đỏ. Có bao nhiêu cách chọn ra $4$ viên bi sao cho phải có $2$ viên bi màu xanh và $3$ viên bi màu đỏ?
	\choice
	{$132$}
	{$3003$}
	{$450$}
	{\True $1200$}
	\loigiai{
	Số cách chọn ra $2$ viên bi màu xanh từ $5$ viên bi màu xanh là $\mathrm{C}_{5}^{2}$ cách.\\
	Số cách chọn ra $3$ viên bi đỏ từ $10$ viên bi đỏ là $\mathrm{C}_{10}^{3}$ cách.\\
	Vậy có 	$\mathrm{C}_{5}^{2}\cdot \mathrm{C}_{10}^{3}=1200$ cách.
	}
\end{ex}

\begin{ex}%[0D2B3-1]%[Dự án đề kiểm tra HKII NH22-23- Huỳnh Quy]%[THPT Quế Sơn, Quảng Nam]%[cau 14]
	Mệnh đề nào dưới đây đúng?
	\choice
	{\True $(x+3)^{4}=x^{4}+12x^{3}+54x^{2}+108x+81$}
	{$(x+3)^{4}+12x^{3}+54x^{2}+108x+324$}
	{$(x+3)^{4}=x^{4}+12x^{3}+54x^{2}+12x+81$}
	{$(x+3)^{4}=x^{4}+108x^{3}+54x^{2}+12x+81$}
	\loigiai{
	Ta có
	\begin{eqnarray*}
	(x+3)^{4}&=&\mathrm{C}_{4}^{0}x^{4}+\mathrm{C}_{4}^{1}x^3\cdot 3+\mathrm{C}_{4}^{2}x^{2}\cdot 3^{2}+\mathrm{C}_{4}^{3}x\cdot 3^{3}+\mathrm{C}_{4}^{4}3^{4}\\
	&=&x^{4}+12x^{3}+54x^{2}+108x+81.\\
	\end{eqnarray*}	
	}
\end{ex}

\begin{ex}%[0D2B3-2]%[Dự án đề kiểm tra HKII NH22-23- Huỳnh Quy]%[THPT Quế Sơn, Quảng Nam]
	Hệ số của số hạng chứa $x^{3}$ trong khai triển $(1-2x)^{5}$ bằng:
	\choice
	{$-56$}
	{\True $-80$}
	{$80$}
	{$56$}
	\loigiai{
	Số hạng tổng quát của khai triển là $\mathrm{C}_{5}^{k}1^{5-k}\cdot(-2x)^{k}=\mathrm{C}_{5}^{k}(-2)^{k}\cdot x^{k}$.\\
	Số hạng này chứa $x^{3}$ khi và chỉ khi $k=3$.\\
	Hệ số của $x^{3}$ là $\mathrm{C}_{5}^{3}\cdot (-2)^{3}=-80$.	
	}
\end{ex}


\Closesolutionfile{ans}
%\begin{center}
%	\textbf{ĐÁP ÁN}
%	\inputansbox{10}{ans/ans}	
%\end{center}
\begin{center}
	\textbf{PHẦN 2 - TỰ LUẬN}
\end{center}
\begin{bt}%[0D2B2-5]%[Dự án đề kiểm tra HKII NH22-23- Huỳnh Quy]%[THPT Quế Sơn, Quảng Nam]
	Cho tập hợp $X=\{0;1;2;3;4;5;6\}$. Có thể lập được bao nhiêu số tự nhiên gồm $4$ chữ số khác nhau mà mỗi chữ số được lấy từ tập $X$?
	\loigiai{
	Gọi số tự nhiên có $4$ chữ số khác nhau là $\overline{abcd}$.
	\begin{itemize}
		\item Có $6$ cách chọn $a$
		\item Có $6$ cách chọn $b$
		\item Có $5$ cách chọn $c$
		\item Có $4$ cách chọn $d$		
	\end{itemize}
	(Hoặc chọn $b$, $c$, $d$ thì có $\mathrm{A}_{6}^{3}$ cách chọn).	\\
	Vậy có $6\cdot 6\cdot 5\cdot 4=720$ (số).
}
\end{bt}

\begin{bt}%[0D4B1-2]%[Dự án đề kiểm tra HKII NH22-23- Huỳnh Quy]%[THPT Quế Sơn, Quảng Nam]
	Tìm tất cả các giá trị của tham số $m$ để phương trình $x^2+mx+m+8=0$ có hai nghiệm phân biệt.
	\loigiai{
	Phương trình đã cho có $2$ nghiệm phân biệt khi và chỉ khi $\Delta>0$
	\begin{eqnarray*}
	&\Leftrightarrow&m^2-4(m+8)>0\\
	&\Leftrightarrow&m^2-4m-32>0\\
	&\Leftrightarrow&\hoac{&m<-4\\&m>8.}
	\end{eqnarray*}
	Vậy tập $m$ cần tìm là $S=(-\infty;-4)\cup(8;+\infty)$.}
\end{bt}

\begin{bt}%[0H4K1-2]%[Dự án đề kiểm tra HKII NH22-23- Huỳnh Quy]%[THPT Quế Sơn, Quảng Nam]
	Trong mặt phẳng $Oxy$, cho đường tròn $(C)\colon (x-3)^{2}+(y-4)^{2}=36$ và điểm $P(-3;-2)$. Từ điểm $P$ kẻ các tiếp tuyến $PM$ và $PN$ tới đường tròn $(C)$, với $M$, $N$ là các tiếp điểm. Viết phương trình tổng quát của đường thẳng $MN$.
	\loigiai{
		\immini{
	Đường tròn	$(C)$ có tâm $I(3;4)$ và có bán kính $R=6$.\\
	Ta có $MN\perp IP$ nên đường thẳng $MN$ có véc-tơ pháp tuyến là $\vec{IP}=(-6;-6)$ hay $MN$ có VTPT $\vec{n}=(1;1)$.\\
	Ta có $IM=R=6$, $PI=6\sqrt{2}$, ta suy ra tứ giác $IMPN$ là hình vuông. Gọi $K$ là trung điểm $IP$ $\Rightarrow K(0;1)$. Ta có $MN$ qua $K(0;1)$.\\
	Phương trình tổng quát của đường thẳng $MN$ là
	\[
	1\cdot (x-0)+1\cdot(y-1)=0\Leftrightarrow x+y-1=0.
	\]
}{\begin{tikzpicture}[scale=1, line join=round, line cap=round]
	\def\a{2.5}
	\coordinate (I) at (0,0);
	\draw (I) circle (\a);
	\coordinate (M) at ($(I)+(45:\a)$);
	\coordinate (N) at ($(I)+(-45:\a)$);
	\path (I)--(M)--([turn]-90:\a) coordinate (Mt);
	\draw (I)--(M)--(Mt);
	\path (I)--(N)--([turn]90:\a) coordinate (Nb);
	\draw (I)--(N)--(Nb);
	\coordinate (P) at (intersection of M--Mt and N--Nb);
	\draw (P)--(M) (P)--(N) (M)--(N);
	\coordinate (K) at (intersection of I--P and M--N);
	\draw (I)--(P);
	\foreach \x in {P,M,N}{
		\draw[fill=red] (\x) circle (1pt) node[right]{$\x$};	
	}		
	\foreach \x in {I}{
		\draw[fill=red] (\x) circle (1pt) node[left]{$\x$};	
	}
	\foreach \x in {K}{
		\draw[fill=red] (\x) circle (1pt) node[below right]{$\x$};	
	}
	\draw pic[draw=blue,angle radius=2mm]
	{right angle=I--M--P} 
	pic[draw=blue,angle radius=2mm] {right angle=I--N--P}
	pic[draw=blue,angle radius=2mm] {right angle=M--K--P}
	;
	\end{tikzpicture}}
	}
\end{bt}

\begin{bt}%[0D2K2-2]%[Dự án đề kiểm tra HKII NH22-23- Huỳnh Quy]%[THPT Quế Sơn, Quảng Nam]
	Đội thanh niên xung kích nhà trường gồm $24$ học sinh có $7$ học sinh khối $12$, $8$ học sinh khối $11$ và $9$ học sinh khối $10$. Lần này, nhà trường cần chọn ra $9$ học sinh trong đội xung kích để lao động dọn vệ sinh phòng chống dịch Covid-19, hỏi có bao nhiêu cách chọn sao cho
	\begin{enumerate}
		\item Mỗi khối có đúng $3$ học sinh.
		\item Mỗi khối có ít nhất $1$ học sinh.
	\end{enumerate}
	\loigiai{
	\begin{enumerate}
		\item Ta có
			\begin{itemize}
				\item Chọn $3$ học sinh khối $12$ có $\mathrm{C}_{7}^{3}$ cách.
				\item Chọn $3$ học sinh khối $11$ có $\mathrm{C}_{8}^{3}$ cách.
				\item Chọn $3$ học sinh khối $10$ có $\mathrm{C}_{9}^{3}$ cách.
			\end{itemize}
		Vậy có $\mathrm{C}_{7}^{3}\cdot\mathrm{C}_{8}^{3}\cdot\mathrm{C}_{9}^{3}=164640$ cách.
		\item Chọn ngẫu nhiên $9$ học sinh thì có $\mathrm{C}_{24}^{9}$ cách.\\
		Trong đó số cách chọn $9$ học sinh không đủ $3$ khối gồm $4$ trường hợp:
		\begin{itemize}
			\item Chọn $9$ học sinh khối $10$ có $\mathrm{C}_{9}^{9}$ cách.
			\item Chọn $9$ học sinh gồm khối $10$ và khối $11$ là $\mathrm{C}_{17}^{9}-\mathrm{C}_{9}^{9}=243094$ cách.
			\item Chọn $9$ học sinh gồm khối $11$ và khối $12$ là $\mathrm{C}_{15}^{9}=5005$ cách.
			\item Chọn $9$ học sinh gồm khối $10$ và khối $12$ là $\mathrm{C}_{16}^{9}-\mathrm{C}_{9}^{9}=11439$ cách.
		\end{itemize}
	Vậy có $\mathrm{C}_{24}^{9}-1-24309-5005-11439=1266750$ cách.
	\end{enumerate}	
	
	}
\end{bt}

