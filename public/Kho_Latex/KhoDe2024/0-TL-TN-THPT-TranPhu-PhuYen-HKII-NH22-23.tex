
\de{ĐỀ THI HỌC KỲ II NĂM HỌC 2022-2023}{THPT Trần Phú - Phú Yên}
\begin{center}
	\textbf{PHẦN 1 - TRẮC NGHIỆM}
\end{center}
\Opensolutionfile{ans}[ans/ans]
\begin{ex}%Câu 1%[0D3B2-1]
	Gieo 2 con xúc xắc cân đối và đồng chất. Xác suất để tổng số chấm xuất hiện trên hai mặt của 2 con xúc xắc đó không vượt quá 4 là?
	\choice
	{$\dfrac{2}{9}$}
	{$\dfrac{5}{18}$}
	{$\dfrac{1}{6}$}
	{$\dfrac{2}{3}$}
	\loigiai{
		Số phần tử không gian mẫu là $ n\left(\Omega\right)=6\cdot 6=36  $.\\
		Gọi $ A $ là biến cố ~\lq\lq tổng số chấm xuất hiện trên hai mặt của $ 2 $ con xúc sắc không vượt quá 4.\rq\rq\\
		Suy ra $ A=\{(1,1);(1,2);(1,3);(2,1);(2;2);(3,1)\} $.\\
		Do đó $ n(A)=5 $.\\
		Xác suất cần tìm là $ P=\dfrac{1}{6} $.
	}
\end{ex}

\begin{ex}%Câu 2%[0D2B2-2]
	Một lớp 11 có 5 học sinh nam và 3 học sinh nữ học giỏi Toán. Giáo viên chọn $ 4 $ học sinh để dự thi học sinh giỏi Toán cấp trường. Xác suất để có đúng $ 3 $ học sinh nam được chọn bằng bao nhiêu?
	\choice
	{\True $\dfrac{3}{7}$}
	{$\dfrac{5}{7}$}
	{$\dfrac{6}{7}$}
	{$\dfrac{4}{7}$}
	\loigiai{
		Số phần tử không gian mẫu $ n\left(\Omega\right)=\mathrm{C}^4_8= 70 $.\\
		Số cách chọn 4 bạn có đúng $ 3 $ nam là $ \mathrm{C}^3_5\cdot \mathrm{C}^1_3 =30$.\\
		Xác suất cần tìm là $ P=\dfrac{3}{7} $.
	}
\end{ex}

\begin{ex}%Câu 3%[0D1Y3-4]
	Một cửa hàng bán áo sơ mi thống kê số lượng áo bán ra trong tháng $ 1 $ như bảng sau.
	\begin{center}
		\begin{tabular}{|c|l|l|l|l|l|l|}
		\hline Cơ áo & 37 & 38 & 39 & 40 & 41 & 42 \\
		\hline Số lượng & 35 & 42 & 50 & 38 & 32 & 48 \\
		\hline
	\end{tabular}
	\end{center}
	Mốt của mẫu số liệu trên bằng? 
	\choice
	{$ 42  $}
	{$ 39  $}
	{$ 41  $}
	{\True $ 50 $}
	\loigiai{
		Mốt của mẫu số liệu trên là $ 50 $.
	}
\end{ex}

\begin{ex}%Câu 4%[0H4Y3-5]
	Cho parabol $(P)\colon  y^2=14 x$ có đường chuẩn $\Delta$ là
	\choice
	{$\Delta \colon y=\dfrac{7}{2}$}
	{$\Delta \colon x=\dfrac{7}{2}$}
	{\True $\Delta \colon x=-\dfrac{7}{2}$}
	{$\Delta \colon y=-\dfrac{7}{2}$}
	\loigiai{
		Ta có tham số tiêu $ p=7 $, do đó phương trình đường chuẩn là $ x=-\dfrac{7}{2} $.
	}
\end{ex}

\begin{ex}%Câu 5%[0D1B4-3]
	Cho dãy số liệu $1 ; 3 ; 3 ; 6 ; 8 ; 9$. Độ lệch chuẩn của dãy trên bằng bao nhiêu?
	\choice
	{$\dfrac{\sqrt{26}}{3}$}
	{\True $\dfrac{5}{\sqrt{3}}$}
	{$\dfrac{24}{3}$}
	{$\sqrt{\dfrac{24}{3}}$}
	\loigiai{
		Giá trị trung bình của mẫu số liệu $ \overline{x}=\dfrac{1+3+3+6+8+9}{6}=5$.\\
		Độ lệch chuẩn $ s=\sqrt{\dfrac{(1-5)^2+2(3-5)^2+(6-5)^2+(8-5)^2+(9-5)^2}{6}}=\dfrac{5\sqrt{3}}{3} $.
	}
\end{ex}

\begin{ex}%Câu 6%[0D3B2-3]
	Tìm một đội gồm $ 5 $ nam và $ 8 $ nữ. Lập một nhóm gồm $ 4 $ người hát tốp ca, tính xác suất để trong $ 4 $ người được chọn có ít nhất $ 3 $ nữ.
	\choice
	{$\dfrac{56}{143}$}
	{$\dfrac{87}{143}$}
	{\True $\dfrac{70}{143}$}
	{$\dfrac{73}{143}$}
	\loigiai{
		Số phần tử của không gian mẫu $ \mathrm{C}^4_{13}=715 $.\\
		Số cách chọn $ 4 $ người có ít nhất $ 3 $ nữ là $ \mathrm{C}^3_8\cdot \mathrm{C}^1_5+\mathrm{C}^4_8=350$.\\
		Vậy xác suất cần tìm là $ P=\dfrac{70}{143} $.
	}
\end{ex}

\begin{ex}%Câu 7%[0D3B2-1]
	Gieo một con xúc xắc cân đối và đồng chất $ 2 $ lần. Xác suất để không có lần nào xuất hiện mặt $ 6 $ chấm?
	\choice
	{$\dfrac{25}{36}$}
	{$\dfrac{12}{36}$}
	{$\dfrac{6}{36}$}
	{$\dfrac{8}{36}$}
	\loigiai{
	Số phần tử của không gian mẫu là $ n\left(\Omega\right) =6\cdot 6=36 $.\\
	Số lần gieo hai con xúc xắc không có lần nào xuất hiện mặt $ 6 $ chấm là $ 5\cdot 5=25 $.\\
	Do đó xác suất cần tìm là $ P= \dfrac{25 }{36}$.
	}
\end{ex}

\begin{ex}%Câu 8%[0D3Y2-1]
	Tung $ 1 $ đồng xu cân đối và đồng chất $ 3 $ lần. Xác suất để có đúng $ 2 $ lần tung xuất hiện mặt $S$ là
	\choice
	{$\dfrac{1}{2}$}
	{$\dfrac{1}{4}$}
	{$\dfrac{3}{4}$}
	{\True $\dfrac{3}{8}$}
	\loigiai{
		Số phần tử của không gian mẫu $ n\left(\Omega\right)  =2^3=8$.\\
		Gọi $ A $ là biến coos~\lq\lq có đúng $ 2 $ lần tung xuất hiện mặt sấp \rq\rq.\\
		Suy ra $ A=\{SSN; SNS; NSS\} $.\\
		Do đó xác xuất cần  tìm là $ P(A)=\dfrac{3}{8} $.
	}
\end{ex}

\begin{ex}%Câu 9%[0D1B4-1]
	Tìm tứ phân vị của mẫu số liệu  $\begin{array}{lllllll}45 & 92 & 68 & 84 & 10 & 98 & 127\end{array}$.
	\choice
	{$Q_1=10 ; Q_2=84 ; Q_3=127$}
	{\True $Q_1=45$; $Q_2=84$; $Q_3=98$}
	{$Q_1=45 ; Q_2=56,5 ; Q_3=95$}
	{$Q_1=45 ; Q_2=88 ; Q_3=98$}
	\loigiai{
		Xắp xếp mẫu số liệu theo thứ tự không giảm  $\begin{array}{lllllll}10&45& 68&  84 & 92  &  98 & 127\end{array}$.\\
		Khi đó $ Q_2=84 $, $ Q_1=45 $, $ Q_3=98 $.
	}
\end{ex}

\begin{ex}%Câu 10%[0H4Y3-1]
	Cho đường elip có phương trình chính tắc $(E)\colon\dfrac{x^2}{4}+\dfrac{y^2}{3}=1$. Một tiêu điểm của elip đó là
	\choice
	{$F_2(0 ; \sqrt{3})$}
	{$F_2(\sqrt{3}; 0)$}
	{\True $F_2(1 ; 0)$}
	{$F_2(2 ; 0)$}
	\loigiai{
		Ta có $ a^2=4 $, $ b^2=3 $, suy ra $ c=\sqrt{a^2-b^2}=1 $. Do đó tiêu điểm của elip là $ F_{1,2}=\left(\pm 1;0\right)  $.
	}
\end{ex}

\begin{ex}%Câu 11%[0H4Y2-2]
	Phương trình nào sau đây là phương trình đường tròn?
	\choice
	{\True $x^2+y^2-x=0$}
	{$x^2-y^2-2 x+2 y-1=0$}
	{$x^2+y^2+2 x y-1=0$}
	{$x^2+y^2-x-y+9=0$}
	\loigiai{
	Phương trình $x^2+y^2-x=0$ là phương trình đường tròn vì $ a^2+b^2-c>0 $.
	}
\end{ex}

\begin{ex}%Câu 12%[0D3B2-1]
	Tung một đồng xu cân đối và đồng chất hai lần liên tiếp. Tính xác suất để cả hai lần tung đều xuất hiện mặt ngửa.
	\choice
	{$\dfrac{1}{3}$}
	{$\dfrac{1}{2}$}
	{$\dfrac{3}{4}$}
	{\True $\dfrac{1}{4}$}
	\loigiai{
		Số phần tử của không gian mẫu $ n(\Omega)=4 $.\\
		Biế cố hai lần đều xuất hiện mặt ngửa $ A=\{NN\} \Rightarrow n\left(A\right) =1$.\\
		Do đó xác suất cần tìm là $ P(A)=\dfrac{1}{4} $.
	}
\end{ex}

\begin{ex}%Câu 13%[0D3B2-1]
	Tung một đồng xu cân đối và đồng chất ba lần liên tiếp. Tính xác suất để có ít nhất một lần tung xuất hiện mặt sấp.
	\choice
	{\True $\dfrac{7}{8}$}
	{$\dfrac{1}{4}$}
	{$\dfrac{1}{8}$}
	{$\dfrac{3}{4}$}
	\loigiai{
		Số phần tử của không gian mẫu $ n\left(\Omega\right)  =2^3=8$.\\
	Gọi $ A $ là biến cố~\lq\lq có ít nhất một lần tung xuất hiện mặt sấp \rq\rq.\\
	Suy ra $ \overline{A}$ \lq\lq không có lần nào xuất hiện mặt sấp, suy ra $ \overline{A} =\{NNN\} $ và $ n\left(\overline A\right)  =1$.\\
	Do đó xác xuất cần  tìm là $ P(A)=1-P(\overline{A})=\dfrac{7}{8} $.
	}
\end{ex}

\begin{ex}%Câu 14%[0H4B1-4]
	Trong mặt phẳng $Oxy$, tính góc giữa hai đường thẳng $\Delta_1\colon x-2 y-2=0$ và $\Delta_2\colon -x-3 y-2=0$. Góc giữa hai đường thẳng $\Delta_1$ và $\Delta_2$ bằng
	\choice
	{$60^{\circ}$}
	{$135^{\circ}$}
	{$30^{\circ}$}
	{$45^{\circ}$}
	\loigiai{
		Véc-tơ pháp tuyến của hai đường thẳng $ \Delta_1 $ và $ \Delta_2 $ lần lượt là 
		$ \overrightarrow{n}_1=(1;-2) $ và $ \overrightarrow{n}_2 =(-1;-3)$.\\
		Gọi $ \varphi $ là góc giữa hai đường thẳng  $ \Delta_1 $ và $ \Delta_2 $.\\
		Ta có $ \cos \varphi =\left| \dfrac{\overrightarrow{n}_1.\overrightarrow{n}_2}{|\overrightarrow{n}_1|\cdot |\overrightarrow{n}_2|} \right | =\dfrac{5}{\sqrt{5}\cdot \sqrt{10}}= \dfrac{1}{\sqrt{2}}\Rightarrow \varphi =45^\circ$.
	}
\end{ex}

\begin{ex}%Câu 15%[0D1Y3-2]
	Cho dãy số liệu $1 ; 2 ; 7 ; 5 ; 10 ; 9 ; 8$. Số trung vị của dãy trên bằng bao nhiêu?
	\choice
	{$ 7  $}
	{\True $ 6  $}
	{$ 8  $}
	{$ 2 $ }
	\loigiai{
		Xắp xếp mẫu số liệu theo thứ tự không giảm $1 ; 2 ; 5 ; 7 ; 8 ; 9 ; 10$.\\
		Do đó trung vị của mẫu số liệu $ M_e=\dfrac{5+7}{2}=6$.
	}
\end{ex}

\begin{ex}%Câu 16%[0H4B2-3]
	Cho đường tròn $(C)\colon  x^2+y^2-3 x-y=0$. Phương trình tiếp tuyến của $(C)$ tại $M(1 ;-1)$ là:
	\choice
	{$x-3 y-2=0$}
	{\True $x+3 y+2=0$}
	{$x-3 y+2=0$}
	{$x+3 y-2=0$}
	\loigiai{
		Đường tròn $ (C) $ có tâm $ I\left (\dfrac{3}{2};\dfrac{1}{2}\right ) $.\\
		Ta có $ \overrightarrow{MI}=\left (\dfrac{1}{2};\dfrac{3}{2}\right ) =\dfrac{1}{2}(1;3)$.\\
		Do đó phương trình tiếp tuyến của đường tròn $ (C) $ tại $ M $ là $ 1(x-1)+3(y+1)=0\Leftrightarrow x+3y+2=0 $.
	}
\end{ex}

\begin{ex}%Câu 17%[0H4Y1-3]
	Trong mặt phẳng tọa độ, vị trí tương đối của hai đường thẳng sau: $d_1: x-2 y+1=0$ và $d_2:-3 x+6 y-10=0$ là
	\choice
	{Song song}
	{Cắt nhau nhưng không vuông góc nhau}
	{Trùng nhau}
	{Vuông góc với nhau}
	\loigiai{
		Ta có $ \dfrac{1}{-3}=\dfrac{-2}{6}\ne \dfrac{1}{-10} $.\\
		Do đó hai đường thẳng $ d_1 $ và $ d_2 $ song song.
	}
\end{ex}

\begin{ex}%Câu 18%[0D1Y3-1]
	Lớp $ 10C $ của một trường trung học phổ thông có điểm thi môn Toán được cho dưới bảng sau
	\begin{center}
		
	\begin{tabular}{|l|c|c|c|c|c|c|}
		\hline Điểm thi & 5 & 6 & 7 & 8 & 9 & 10 \\
		\hline Tần số & 5 & 7 & 12 & 14 & 3 & 4 \\
		\hline
	\end{tabular}

\end{center}
	Tính điểm trung bình cộng môn Toán của lớp $10C$ (làm tròn đến hàng phần mười)?
	\choice
	{$ 8  $}
	{$ 7,5 $ }
	{$ 7,6  $}
	{\True $ 7,3  $}
	\loigiai{
		Điểm trung bình của lớp $ 10 C $ là
		$ \overline{x}=\dfrac{5\cdot 5+7\cdot 6+12\cdot 7+14\cdot 8+3\cdot 9+4\cdot10}{45} \approx 7,3$.
	}
\end{ex}

\begin{ex}%Câu 19%[0D1Y3-2]
	Cân nặng $(\mathrm{kg})$ của một nhóm học sinh lớp 10 cho bởi số liệu sau \linebreak 
$44; 46; 46;	48;	52; 60; 75; 76; 76; 78$.
	Tìm số trung vị? 
	\choice
	{$ 75 $ }
	{$ 52  $}
	{\True $ 56 $ }
	{$ 60  $}
	\loigiai{
		Số trung vị của mẫu số liệu trên là $ M_e =\dfrac{52+60}{2}=56$.
	}
\end{ex}

\begin{ex}%Câu 20%[0D1B4-3]
	Chiều dài (đơn vị feet) của 7 con cá voi trưởng thành được cho trong bảng sau
	\begin{center} $\begin{array}{lllllll}48 & 53 & 51 & 31 & 53 & 112 & 52 .\end{array}$
		\end{center}
	Tính phương sai của mẫu số liệu trên (làm tròn đến hàng phần chục)? 
	\choice
	{$s^2=550,2$}
	{$s^2=250,4$}
	{$s^2=553,6$}
	{$s^2=450,3$}
	\loigiai{
		Chiều dài trung bình của cá voi là $ \overline{x} = \dfrac{48+53+51+31+53+112+52}{7}\approx 56,14$.\\
		Phương sai của mẫu số liệu
		$$ s^2=\dfrac{(48-\overline{x} )^2+2\cdot (53-\overline{x} )^2+(51-\overline{x} )^2+(31-\overline{x} )^2+(112-\overline{x} )^2+(52-\overline{x} )^2}{7}\approx553,6.$$
	}
\end{ex}

\begin{ex}%Câu 21%[0D1Y4-1]
	Cho mẫu số liệu về chiều cao (cm) của 5 em học sinh là $156 ; 170 ; 153 ; 160 ; 175$. Khoảng biến thiên của mẫu số liệu là 
	\choice
	{\True $ 22  $}
	{$ 19  $}
	{$ 15  $}
	{$ 175 $}
	\loigiai{
		Khoảng biến thiên $ R=175-153=22 $.
	}
\end{ex}

\begin{ex}%Câu 22%[0H4Y1-5]
	Trong mặt phẳng với hệ tọa độ $Oxy$, khoảng cách từ điểm $A(1 ; 3)$ đến đường thẳng $3 x+y+4=0$ là
	\choice
	{$12 \sqrt{3}$}
	{\True $\sqrt{10}$}
	{$ 18  $}
	{$ 39  $}
	\loigiai{
	Khoảng cách từ $ A $ đến đường thẳng $ 3x+y+4=0 $ là $ \mathrm{d}=\dfrac{\left| 3\cdot 1+3+4\right | }{\sqrt{10}}=\sqrt{10} $.
	}
\end{ex}

\begin{ex}%Câu 23%[0H4Y2-1]
	Đường tròn $(C)\colon (x+1)^2+(y+2)^2=9$ có tâm $I$ là
	\choice
	{\True $I(-1 ;-2)$}
	{$I(1 ;-2)$}
	{$I(1 ; 2)$}
	{$I(-1 ; 2)$}
	\loigiai{
		Đường tròn $ (C) $ có tâm $ I(-1;-2) $.
	}
\end{ex}

\begin{ex}%Câu 24%[0D1Y3-1]
	Cho dãy số liệu $1 ; 3 ; 4 ; 6 ; 8 ; 9 ; 11$. Số trung bình cộng của dãy số liệu trên? 
	\choice
	{$ 11  $}
	{\True $ 6 $ }
	{$ 9  $}
	{$ 5  $}
	\loigiai{
		Số trung bình cộng của dãy số liệu trên là $ \overline{x}=\dfrac{1+3+4+6+8+9+11}{7}= 6$.
	}
\end{ex}

\begin{ex}%Câu 25%[0D3B2-4]
	Một hộp đựng $ 12 $ bóng đèn trong đó có $ 4 $ bóng hỏng. Chọn ngẫu nhiên $ 3 $ bóng. Xác suất của biến cố $A\colon$   \lq\lq \textbf{Không} có bóng đèn nào hỏng\rq\rq~ là?
	\choice
	{$\dfrac{28}{55}$}
	{$\dfrac{8}{55}$}
	{$\dfrac{7}{55}$}
	{\True $\dfrac{14}{55}$}
	\loigiai{
		Số phần tử không gian mẫu là $ n(\Omega) =\mathrm{C}^3_{12}=220$.\\
		Số cách chọn ra ba bóng không có bóng hỏng là $ n(A)=\mathrm{C}^3_{8}=56$.\\
		Xác suất cần tìm là $ P(A)=\dfrac{14}{55} $.
	}
\end{ex}

\begin{ex}%Câu 26%[0D1Y3-2]
	Điểm thi môn Toán của lớp 10B của một trường trung học phổ thông cho bởi bảng
	\begin{center}
	\begin{tabular}{|l|l|l|l|l|}
		\hline Điểm thi & 6 & 7 & 8 & 9 \\
		\hline Tần số & 9 & 18 & 12 & 6 \\
		\hline
	\end{tabular} \end{center}
	Tìm số trung vị.
	\choice
	{$ 15  $}
	{$ 7,5  $}
	{\True $ 7  $}
	{$ 8  $}
	\loigiai{
		Xắp xếp mẫu số liệu theo thứ tự không giảm, suy ra trung vị của mẫu số liệu là $ 7 $.
	}
\end{ex}

\begin{ex}%Câu 27%[0D1B4-1]
	Mẫu số liệu thống kê chiều cao $(\mathrm{m})$ của 15 cây bạch đàn là:
	$$
	\begin{array}{lllllllllllllll}
		6,3 & 6,6 & 7,2 & 7,5 & 7,5 & 7,6 & 7,7 & 7,8 & 7,9 & 8,2 & 8,3 & 8,7 & 8,8 & 8,9 & 9,0
	\end{array}
	$$
	Khoảng tứ phân vị của mẫu số liệu là.
	\choice
	{\True $\Delta_Q=1,2(\mathrm{~m})$}
	{$\Delta_Q=1,4(\mathrm{~m})$}
	{$\Delta_Q=1,3(\mathrm{~m})$}
	{$\Delta_Q=1(\mathrm{~m})$}
	\loigiai{
		Tứ phân vị $ Q_2 = 7,8$, $ Q_1=7,5 $ và $ Q_3=8,7 $.\\
		Do đó khoảng tứ phân vị là $ \Delta Q=Q_3-Q_1=1,2 $.
	}
\end{ex}

\begin{ex}%Câu 28%[0D3Y1-1]
	Gieo một đồng xu cân đối và đồng chất 2 lần. Số phần tử của không gian mẫu là
	\choice
	{$ 8  $}
	{$ 6  $}
 	{\True $ 4  $}
	{$ 2 $ }
	\loigiai{
		Số phần tử không gian mẫu $ n\left(\Omega\right) =4  $.
	}
\end{ex}

\begin{ex}%Câu 29%[0D1Y4-2]
	Những giá trị nào bất thường trong mẫu số liệu thống kê sau? 
	 $$3 ; 27 ; 28 ; 29 ; 30 ; 31 ; 32 ; 33 ; 47 ; 48$$
	\choice
	{$3 ; 48$}
	{$3 ; 27 ; 48$}
	{$47 ; 48$}
	{\True $3 ; 47 ; 48$}
	\loigiai{
		Ta có $ Q_2= \dfrac{30+31}{2}=\dfrac{61}{2}$, $ Q_1=28$, $ Q_3=33$.\\
		Do đó khoảng tứ phân vị là $ \Delta_Q= 5$.\\
		Ta có $ Q_1-1,5\cdot \Delta_Q= \dfrac{41}{2}=20,5$ và $ Q_3+1,5\cdot \Delta_Q= 40,5$.\\
		Do đó giá trị bất thường của mẫu số liệu trên là $ 3 $, $ 47 $, $ 48 $.
	}
\end{ex}

\begin{ex}%Câu 30%[0D3Y1-1]
	Một hộp có 5 quả cầu được đánh số từ $ 1 $ đến $ 5  $. Lấy ngẫu nhiên từ hộp đó liên tiếp $ 2 $ lần, mỗi lần một quả và xếp thành hàng ngang. Số phần tử của không gian mẫu là
	\choice
	{\True $ 20  $}
	{$ 10  $}
	{$ 120  $}
	{$ 2  $}
	\loigiai{
		Số phần tử của không gian mẫu là $ n\left(\Omega\right) =\mathrm{A}^2_5=20 $.
	}
\end{ex}

\begin{ex}%Câu 31%[0D3B2-9]
	Cho $\mathrm{A}$ là một biến cố liên quan phép thử $\mathrm{T}$. Mệnh đề nào sau đây là mệnh đề \textbf{sai}?
	\choice
	{$P(A)=0 \Leftrightarrow A=\Omega$}
	{$P(A) \leq 1$}
	{$P(A) \geq 0$}
	{$P(A)=1-P(\bar{A})$}
	\loigiai{
	Mệnh đề \lq\lq $P(A)=0 \Leftrightarrow A=\Omega$\rq\rq ~ là một mệnh đề sai.
	}
\end{ex}

\begin{ex}%Câu 32%[0H4Y3-5]
	Phương trình nào sau đây là phương trình chính tắc của đường hypebol?
	\choice
	{$\dfrac{x^2}{7}+\dfrac{y^2}{2}=1$}
	{$\dfrac{x^2}{4}+\dfrac{y^2}{7}=1$}
	{$\dfrac{x^2}{4}+\dfrac{y^2}{5}=1$}
	{\True $\dfrac{x^2}{7}-\dfrac{y^2}{2}=1$}
	\loigiai{
		Phương trình chính tắc của hypebol là $\dfrac{x^2}{7}-\dfrac{y^2}{2}=1$.
	}
\end{ex}

\begin{ex}%Câu 33%[0H4Y2-1]
	Đường tròn $x^2+y^2-6 x-8 y=0$ có bán kính bằng bao nhiêu?
	\choice
	{$ 25  $}
	{\True $ 5  $}
	{$\sqrt{10}$}
	{$ 10 $}
	\loigiai{
		Ta có $x^2+y^2-6 x-8 y=0\Leftrightarrow (x-3)^2+(y-4)^2=25$. Do đó bán kính đường tròn là $ R=5 $.
	}
\end{ex}

\begin{ex}%Câu 34%[0D3B2-1]
	Tung một đồng xu cân đối và đồng chất liên tiếp cho đến khi xuất hiện mặt sấp hoặc cả năm lần ngửa thì dừng lại. Tính xác suất để số lần tung không vượt quá bốn.
	\choice
	{$\dfrac{2}{5}$}
	{\True  $\dfrac{2}{3}$}
	{$\dfrac{3}{5}$}
	{$\dfrac{1}{3}$}
	\loigiai{
		Số phần tử của không gian mẫu $ n(\omega)=\{S; NS; NNS; NNNS;NNNNS;NNNNN\} $.\\
		Xác suất để số lần không vượt quá bốn là $ P=\dfrac{4}{6}=\dfrac{2}{3} $.
	}
\end{ex}

\begin{ex}%Câu 35%[0D3B2-3]
	Sắp xếp $ 12 $ học sinh lớp $ 10 $ gồm có $ 6 $ học sinh nam, $ 6 $ học sinh nữ vào một bàn dài gồm có hai dãy ghế đối diện nhau (mỗi dãy có $ 6 $ ghế) để thảo luận. Tính xác suất để hai học sinh ngồi đối diện và cạnh nhau luôn khác giới?
	\choice
	{$\dfrac{1}{924}$}
	{\True $\dfrac{1}{462}$}
	{$\dfrac{3}{9920}$}
	{$\dfrac{1}{665280}$}
	\loigiai{
		Đánh số vị trí ngồi của hai dãy ghế đối diện nhau như hình vẽ.
		\begin{center}
	\begin{tabular}{|c|c|c|c|c|c|}
		\hline
		1 & 2 & 3 & 4 & 5 & 6 \\
		\hline
		7 & 8 & 9 & 10 & 11 & 12 \\
		\hline
	\end{tabular}
\end{center}
Số phần tử không gian mẫu bằng số cách xếp $ 12  $ học sinh nên ta có $ 12! $ cách.\\
Để xếp hai học sinh ngồi đối diện và cạnh nhau luôn khác giới ta làm như sau:\begin{itemize}
	\item Xếp học sinh ở vị trí $ 1 $ có $ 12 $ cách.
	\item Xếp ở vị trí $ 7 $ có $ 6 $ cách.
	\item Xếp ở vị trí $ 2 $ có $ 5 $ cách.
	\item Xếp ở vị trí $ 8 $ có $ 5 $ cách.
	\item Xếp ở vị trí $ 3 $ có $ 4 $ cách.
	\item Xếp ở vị trí $ 9 $ có $ 4 $ cách.
	\item Xếp ở vị trí $ 4 $ có $ 3 $ cách.
	\item Xếp ở vị trí $ 10 $ có $ 3 $ cách.
	\item Xếp ở vị trí $ 5 $ có $ 2 $ cách.
	\item Xếp ở vị trí $ 11 $ có $ 2 $ cách.
	\item Xếp ở vị trí $ 6 $ có $ 1 $ cách.
	\item Xếp ở vị trí $ 12 $ có $ 1 $ cách.
\end{itemize}
Vậy xác suất cần tìm là $ P=\dfrac{1^2\cdot2^2\cdot3^2\cdot4^2\cdot5^2\cdot6\cdot 12}{12!}= \dfrac{1}{462}$.
	}
\end{ex}
\Closesolutionfile{ans}
%\begin{center}
%	\textbf{ĐÁP ÁN}
%	\inputansbox{10}{ans/ans}	
%\end{center}
\begin{center}
	\textbf{PHẦN 2 - TỰ LUẬN}
\end{center}

\begin{bt}%Câu 36(1.0 điểm ) 
Tung đồng xu cân đối đồng chất hai lần liên tiếp. Cho biến cố $A\colon $   \lq\lq Lần đầu xuất hiện mặt ngửa\rq\rq. Liệt kê các kết quả thuận lợi của biến cố $A$.
	\loigiai{
	Các kết quả thuận lợi của biến cố $ A $ là $ \{NN; NS\} $.
	}
\end{bt}

\begin{bt}%Câu 37%[0H4B2-2](1.0 điểm ) 
Viết phương trình tổng quát của đường tròn có đường kính $A B$ biết $A(-3 ; 2)$; $B(1 ; 4)$.
	\loigiai{
		Tọa độ trung điểm của $ AB $ là $ I(-1;3) $.\\
		Bán kính đường tròn $ R=IA=\sqrt{2^2+1^2}=\sqrt{5} $.\\
		Vậy phương trình đường tròn đường kính $ AB $ có phương trình là $ (x+1)^2+(y-3)^2=5 $.
	}
	
\end{bt}

\begin{bt}%Câu 38%[0D3B2-4](1.0 điểm ) 
Một hộp đựng $ 7 $ bi xanh và $ 6 $ bi đỏ. Chọn ngẫu nhiên $ 5 $ viên bi. Tính xác suất để $ 5 $ bi được chọn có đủ hai màu và số bi xanh nhiều hơn số bi đỏ.
	\loigiai{
		Số phần tử của không gian mẫu là $ n(\Omega) =\mathrm{C}^5_{13}= 1287$.\\
		Gọi $ A $ là biến cố~ \lq\lq  $ 5 $ bi được chọn có đủ hai màu và số bi xanh nhiều hơn số bi đỏ\rq\rq.\\
	Để chọn $ 5 $ bi được chọn có đủ hai màu và số bi xanh nhiều hơn số bi đỏ ta có các trường hợp sau
	\begin{itemize}
		\item [\bf TH1:] $ 3 $ xanh và $ 2 $ đỏ có $ \mathrm{C}^3_7 \cdot\mathrm{C}^2_6 =525$.
		\item [\bf TH2:] $ 4 $ xanh và $ 1 $ đỏ có $ \mathrm{C}^4_7 \cdot\mathrm{C}^1_6 =210$.		
	\end{itemize}
Suy ra $ n\left(A\right)=735  $.\\
Xác suất cần tìm là $ P(A)= \dfrac{245}{429}$.

 
	}
\end{bt}